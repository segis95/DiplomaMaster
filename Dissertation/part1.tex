\chapter{Задача сегментации и виды разметки данных} \label{chapt1}


\section{Постановка задачи. Сильная и слабая разметка данных}
Центральным объектом рассмотрения в данной работе является задача сегментации медицинских изображениях.
При решении задачи {\bf сегментации}  новообразований/поражений/аномалий необходимо для каждого пикселя изображения установить его принадлежность к проблемной области(положительному классу) на снимке. При этом каждый набор изображений может поставляться с некоторого рода дополнительными сведениями, называемыми {\bf разметкой}.

Говорят, что набор изображений $\{M_i\}_{i\in I}\subset \mathcal{M}^{mxn}$ имеет {\bf сильную разметку}, если для каждого пикселя с координатами $(x,y)$ известно, относится ли он к положительному классу или нет. При этом каждому изображению $M_i$ ставится в соответствие битовая маска $B_i$, отражающая данную информацию: $$\{B_i\}_{i\in I} \subset \mathcal{M}^{mxn},$$ где $$\forall i \in I,  \forall x,y: 1\le x \le m, 1\le y \le n, B_i(x,y) \in \{0,1\}.$$

Получение сильной разметки данных в ручном режиме сопряжено со значительными временными и финансовыми затратами: попиксельная разметка изображений в высоком разрешении требует кропотливого труда высококлассного опытного специалиста, а также клинического подтверждения диагноза, к примеру, путём взятия биопсии. Для оценки можно принять, что разметка одного изображения занимает около 5 минут. Тогда разметка 10.000 изображений вручную потребует 50.000 минут или более 100 рабочих дней непрерывного труда без учёта времени на валидацию разметки. В таких условиях целесообразно каким-то образом устранить необходимость наличия сильной разметки. Однако возникает вопрос: информации какого типа будет достаточно для получения сопоставимой по качеству системы сегментации с той, которая получена на основе набора данных с сильной разметкой. Наборы изображений, для которых отсутствует сильная разметка, но имеется иная разметка данных, позволяющая каким-то образом установить наличие либо отсутствие пикселей положительного класса, называются {\bf слабо размеченными}. Примерами слабой разметки может служить:

\begin{itemize}
    \item Наличие классификации(прямое отнесение изображения к одному из двух классов в зависимости от наличия новообразований);
    \item Наличие размеров новообразований если таковые имеются на изображении;
    \item Приблизительная локализация локализация: точка и радиус диаметра окружности около неё;
    \item Наличие прямоугольной рамки, заключающей новообразование(Bounding box);
    \item $\dots$
\end{itemize}

В данной работе речь пойдёт главным образом о первом виде слабой разметки.

В следующем разделе приведём несколько примеров задач сегментации изображений, возникающих в медицинской практике.

\section{Примеры задач сегментации}

\subsection{Гистопатология}

{\bf Задача:} по данным высококачественным изображениям тканей распознать наличие поражений, локализовать их и разбить на кластеры(\ref{fig:histo}).
\\
{\bf Пример работы:} \cite{xu_weakly_2014}


\begin{figure}[ht] 
  \center
  \includegraphics [scale=0.27] {images/histo_1.png}
  \caption{Примеры здоровых(зелёные) и поражённых(красные) раком областей ткани; \cite{xu_weakly_2014}} 
  \label{fig:histo}  
\end{figure}



\subsection{Задачи, связанные с изображениями мозга}

\subsubsection{Выделение белого вещества в мозге}
{\bf Задача:} по данным изображениям МРТ сегментировать участки, представляющие собой белое вещество мозга(\ref{fig:whitematter});
\\
{\bf Пример работы:} \cite{white_matter}


\begin{figure}[ht] 
  \center
  \includegraphics [scale=1.0] {images/white_matter.png}
  \caption{ Пример изображения с сегментацией белого вещества\cite{white_matter} } 
  \label{fig:whitematter}  
\end{figure}


\subsubsection{Сегментация опухолей головного мозга}

{\bf Задача:} детектировать и локализовать злокачественные новообразования (\ref{fig:brats}).

{\bf Пример работы}: \cite{BRATS_winning_2018}
\begin{figure}[ht] 
  \center
  \includegraphics [scale=0.8] {images/brats.png}
  \caption{ Пример изображения мозга с сегментацией новообразования и его составных частей \cite{BRATS_2018} } 
  \label{fig:brats}  
\end{figure}


\subsection{Задачи, связанные с раком молочной железы}

\subsubsection{Сегментация тканей молочной железы}
{\bf Задача:} отделить на снимке более плотные ткани от мягких. Более плотные ткани связаны с повышенным риском развития злокачественных новообразований, однако средняя плотность тканей не является постоянной для всех пациентов(на это влияет в том числе процент жировой ткани), что усложняет задачу выделения плотных областей на снимке(\ref{fig:breast_1r}, \ref{fig:breast_2})
\\
{\bf Пример работы:} \cite{breast_tissue}


\begin{figure}[h] 
  \center
  \includegraphics [scale=1.0] {images/breast_1.png}
  \caption{ Пример сегментации(красная сплошная линия) для пациента с низкой совокупной плотностью тканей\cite{breast_tissue}} 
  \label{fig:breast_1r}  
\end{figure}

\begin{figure}[!htb] 
  \center
  \includegraphics [scale=1.0] {images/breast_2.png}
  \caption{ Пример сегментации(красная сплошная линия) для пациента с высокой совокупной плотностью тканей\cite{breast_tissue}} 
  \label{fig:breast_2}  
\end{figure}
\
\subsubsection{Сегментация злокачественных новообразований}

{\bf Задача:} детектировать и локализовать злокачественные новообразования.(рис. \ref{fig:breast_3})
\\
{\bf Пример работы:} \cite{zhu_mil}

\begin{figure}[h] 
  \center
  \includegraphics [scale=0.8] {images/breast_cancer.png}
  \caption{ Пример сегментации новообразований в молочной железе\cite{zhu_mil} }
  \label{fig:breast_3}  
\end{figure}



\subsection{Задачи, связанные с раком предстательной железы}

{\bf Задача:} сегментировать предстательную железу на изображении, детектировать и локализовать доброкачественные и  злокачественные новообразования(\ref{fig:prostate}).

{\bf Пример работы}: \cite{prostate-cancer}

\begin{figure}[h] 
  \center
  \includegraphics [scale=0.8] {images/prostate.png}
  \caption{ Пример изображения с выделенными {\color{blue} железой}, {\color{green} центральной частью}, {\color{orange} доброкачественным новоообразованием}, {\color{red} злокачественным новоообразованием} \cite{prostate-cancer} } 
  \label{fig:prostate}  
\end{figure}



\subsection{Задачи, связанные с поражениями сетчатки глаза}

{\bf Задача:} на данном изображении сетчатки глаза сегментировать поражения, связанные с диабетичесской ретинопатией(\ref{fig:retina}).

{\bf Пример работы:}\cite{retinopathy}. В данной работе предлагается метод совместного решения трёх задач: постановка диагноза глаукомы, сегментация диска зрительного нерва и сегментация поражений.

\begin{figure}[h] 
  \center
  \includegraphics [scale=0.8] {images/glaucoma.png}
  \caption{ Изображения внутренней поверхности глазного яблока с выделенными аномалиями} 
  \label{fig:retina}  
\end{figure}


Безусловно, приведённый список примеров не является исчерпывающим и не претендует на полноту. Однако он позволяет судить об актуальности объекта настоящего исследования и об активном развитии данного направления работы во всём мире.

