\chapter*{Введение}							% Заголовок
\addcontentsline{toc}{chapter}{Введение}	% Добавляем его в оглавление

Согласно статистическим данным \cite{Stats}, за последние 20 лет число аппаратов магнитно-резонансной томографии(МРТ) кратно увеличилось во многих развитых странах, что  сделало данный вид исследования более доступным и широко распространённым в медицинской практике. В связи с этим появилась возможность собирать базы данных изображений для различных видов диагнозов.
\\
\indent Рост числа проводимых исследований пациентов также неизбежно повлёк за собой пропорциональное увеличение времени, необходимое на их обработу в ручном режиме. Естественным образом возникла задача сокращения времени ручной обработки данных исследования вплоть до его полной автоматизации.
\\
\indent Вместе с тем, сбор и обработка баз изображений c разметкой областей, свидетельствующих о наличии диагноза - дорогостоящий, тяжёлый и кропотливый труд. При составлении баз изображений важны следующие аспекты:

\begin{itemize}
    \item {\bf Уровень квалификации} специалиста, обрабатывающего данные;
    \item {\bf Однотипность} параметров {\bf оборудования}, на котором изображения получены;
    \item {\bf Массовость}: чем больше различных пациентов и различных изображений, тем лучше;
    \item {\bf Достоверность}: в случае онкологических заболеваний, к примеру, может быть окончательно подтверждён лишь посредством взятия биопсии;
    \item {\bf Однородность} выборки: одинаковый формат изображений и аннотации к ним;
\end{itemize}

\\
\indent Современные методы компьютерного зрения в массе своей основаны на применении свёрточных нейронных сетей. Как известно, глубокие архитектуры нейронных сетей имеют миллионы параметров, а следовательно, требуют значительного количества данных для обучения. В случае с задачей {\bf сегментации  новообразований} на медицинских изображениях, оказывается очень затратным получать качественную в упомянутых выше смыслах значительную по размерам(от нескольких тысяч пациентов) выборку данных, поскольку в качестве разметки должна выступать попиксельная сегментация(такая разметка изображений называется {\bf сильной}), требующая многочасового кропотливого труда высококлассного специалиста. С другой стороны, в десятки раз быстрее тот же специалист может прийти к заключению о наличии или отсутствии следов новообразований на изображении без необходимости выделения контура самого новообразования, когда оно присутствует на снимке. Такое заключение называют слабой разметкой изображения. 
\\
\indent В даннной работе исследуется возможность снизить необходимое количества сильно размеченных изображений за счёт использования изображений со слабой разметкой.
