\chapter*{Заключение}						% Заголовок
\addcontentsline{toc}{chapter}{Заключение}	% Добавляем его в оглавление

%% Согласно ГОСТ Р 7.0.11-2011:
%% 5.3.3 В заключении диссертации излагают итоги выполненного исследования, рекомендации, перспективы дальнейшей разработки темы.
%% 9.2.3 В заключении автореферата диссертации излагают итоги данного исследования, рекомендации и перспективы дальнейшей разработки темы.
%% Поэтому имеет смысл сделать эту часть общей и загрузить из одного файла в автореферат и в диссертацию:

\noindent Основные результаты работы заключаются в следующем.
%%% Согласно ГОСТ Р 7.0.11-2011:
%% 5.3.3 В заключении диссертации излагают итоги выполненного исследования, рекомендации, перспективы дальнейшей разработки темы.
%% 9.2.3 В заключении автореферата диссертации излагают итоги данного исследования, рекомендации и перспективы дальнейшей разработки темы.
\begin{enumerate}
  \item На основе анализа \ldots
  \item Численные исследования показали, что \ldots
  \item Математическое моделирование показало \ldots
  \item Для выполнения поставленных задач был создан \ldots
\end{enumerate}


\begin{enumerate}
    \item Был предложен метод решения задачи сегментации с использованием одновременно слабой разметки и сильной разметки относительно небольшой части набора данных;
    \item Произведены многочисленные эксперименты, описывающие свойства полученного метода и его вариаций ;
\end{enumerate}

Результаты данной работы открывают возможность для оптимизации ресурсов при планировании разметки данных: начиная от необходимого количества сильно размеченных изображений, заканчивая возможным смещением усилий по разметке в каком-то опредеённом направлении(к примеру, размечать такие снимки, которые содержат небольшую по относительной площади опухоль, поскольку такие пограничные случаи разрешаются системой сегментации гораздо менее эффективно). 
